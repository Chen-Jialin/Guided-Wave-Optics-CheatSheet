%! TeX program = pdfLaTeX
% Guided Wave Optics CheatSheet
\documentclass[a4paper,10pt]{article}
\usepackage[UTF8]{ctex}
\usepackage[margin=.1in]{geometry}
\usepackage{multicol}
\setlength{\columnseprule}{.2pt}
\usepackage{amsmath, amssymb, amsthm, bm, mathrsfs}
\providecommand{\abs}[1]{\lvert#1\rvert}
\providecommand{\re}{\,\text{Re}\,}
\providecommand{\im}{\,\text{Im}\,}
\providecommand{\arctanh}{\,\text{arctanh}\,}
\usepackage{ulem}
\normalem
\newcommand{\msout}[1]{\text{\sout{\ensuremath{#1}}}}
\newcommand{\muline}[1]{\text{\uline{\ensuremath{#1}}}}
\newcommand{\mdashuline}[1]{\text{\dashuline{\ensuremath{#1}}}}
\newcommand{\mdotuline}[1]{\text{\dotuline{\ensuremath{#1}}}}
\usepackage{soul}
\usepackage{tikz}
\begin{document}
\scriptsize
\begin{multicols}{2}
    \noindent\textbf{麦克斯韦方程组(时域)}:$\nabla\times\bm{E}(\bm{r},t)=-\partial\bm{B}(\bm{r},t)/\partial t$(\textbf{法拉第电磁感应定律})\\
    \indent$\nabla\times\bm{H}(\bm{r},t)=\bm{J}(\bm{r},t)+\partial\bm{D(\bm{r},t)}/\partial t$(\textbf{安培定律})\\
    \indent$\nabla\cdot\bm{B}(\bm{r},t)=0$(\textbf{磁高斯定律}),\qquad$\nabla\cdot\bm{D}(\bm{r},t)=\rho(\bm{r},t)$(\textbf{电高斯/库仑定律})\\
    \indent其中$\bm{E}$-电场强度(V/m),$\bm{H}$-磁场强度(A/m),$\bm{D}$-电位移矢量/电通量密度(C/m$^2$),$\partial\bm{D}/\partial t$-位移电流,$\bm{B}$-磁感应强度/磁通量密度(T,Wb/m$^2$)\\
    \indent无源条件下(下同),自由电流密度$\bm{J}=0$,电荷密度$\rho=0$\\
    \textbf{麦氏方程组(频域,无源)}:$\nabla\times\bm{E}(\bm{r},\omega)=-j\omega\bm{B}(\bm{r},\omega)$,\qquad$\nabla\times\bm{H}(\bm{r},\omega)=j\omega\bm{D}(\bm{r},\omega)$\\
    \indent$\nabla\cdot\bm{B}(\bm{r},\omega)=0$,\qquad\indent$\nabla\cdot\bm{E}(\bm{r},\omega)=0$\\
    \textbf{本构关系}:$\bm{D}=\epsilon_0\bm{E}+\bm{P}\approx(\text{弱场})\epsilon_0(1+\chi)\bm{E}=\epsilon_0\epsilon_r\bm{E}=\epsilon\bm{E}$\\
    \indent$\bm{B}=\mu\bm{H}=\mu_0\mu_r\bm{H}\approx(\text{非磁介质})\mu_0\bm{H}$\\
    \indent其中$\epsilon$-介电常数,真空…$\epsilon_0=8.85\times 10^{-12}$F/m$\approx(36\pi)^{-1}\times 10^{-9}$F/m,$\epsilon_r$-相对…,$\chi$-电极化率,弱场下,电极化强度$\bm{P}=\chi\bm{E}$,$\mu$-磁导率,真空…$\mu_0=4\pi\times 10^{-7}$H/m,对非磁介质(下同),相对…$\mu_r=1$\\
    \textbf{边界条件}:平行界面有$\bm{E}_{1t}=\bm{E}_{2t}$,$\bm{H}_{1t}=\bm{H}_{2t}$,\qquad垂直界面有$D_{1n}=D_{2n}$,$B_{1n}=B_{2n}$\\
    \textbf{亥姆霍兹方程}:$\nabla^2\bm{E}+k^2\bm{E}=0$,$\nabla^2\bm{H}+k^2\bm{H}=0$,其中波矢$\bm{k}=\omega^2\mu\epsilon\hat{k}=\frac{\omega}{v}\hat{k}$,波速$v=1/\sqrt{\mu\epsilon}=1/\sqrt{\mu_0\mu_r\epsilon_0\epsilon_r}\approx c/n$,真空光速$c=1/\sqrt{\epsilon_0\mu_0}$,折射率$n=\sqrt{\mu_r\epsilon_r}\approx\sqrt{\epsilon_r}$\\
    \indent有\textbf{平面波}解$\bm{E}=\bm{E}_0e^{-j\bm{k}\cdot\bm{r}}$,$\bm{H}=\bm{H}_0e^{-j\bm{k}\cdot\bm{r}}$\\
    \textbf{电场,磁场\&波矢的关系}:$\bm{k}\times\bm{E}_0=\omega\mu\bm{H}_0$,$\bm{k}\times\bm{H}_0=-\omega\epsilon\bm{E}_0$,$\bm{E}_0=\sqrt{\mu/\epsilon}\bm{H}_0\times\hat{k}=\eta\bm{H}_0\times\hat{k}$,$\bm{H}_0=\frac{1}{\eta}\hat{k}\times\bm{E}_0$,其中阻抗$\eta=\sqrt{\mu/\epsilon}=\eta_0/n$,真空阻抗$\eta_0=\sqrt{\mu_0/\epsilon_0}$\\
    \textbf{波印廷矢量(能流)}:$\bm{S}=\frac{1}{2}\re[\bm{E}\times\bm{H}^*]=\frac{1}{2\eta}\abs{E_0}^2\hat{k}=\frac{\eta}{2}\abs{H_0}^2\hat{k}$\\
    \textbf{偏振}:电场振动方向,$\bm{E}=\hat{x}E_x+\hat{y}E_y=\hat{x}E_{x0}\cos(kz-\omega t+\phi_x)+\hat{y}E_{y0}\cos(kz-\omega t+\phi_y)$\\
    \indent若$\phi_x=\phi_y+n\pi$,$\bm{E}=(\hat{x}E_{x0}\pm\hat{y}E_{y0})\cos(kz-\omega t+\phi_x)$,线偏\\
    \indent若$\Delta\phi=\phi_y-\phi_x=-\pi/2+2n\pi$,右旋(IEEE标准:逆传播方向看)\\
    \indent若$\Delta\phi=\pi/2+2n\pi$,左旋\\
    \indent$(\frac{E_x}{E_{x0}})^2+(\frac{E_y}{E_{y0}})^2-2\frac{E_x}{E_{x0}}\frac{E_y}{E_{y0}}\cos\Delta\phi=\sin^2\Delta\phi$,其中长轴与$x$轴夹角$\alpha=\arctan 2E_{x0}E_{y0}/(E_{x0}^2-E_{y0}^2)$\\
    \indent若$\alpha=0$,$\Delta\phi=\pm\frac{\pi}{2}$,$(E_x/E_{x0})^2+(E_y/E_{y0})^2=1$,正椭偏,若还$E_{x0}=E_{y0}$,圆偏\\
    \indent若$\Delta\phi=n\pi$,$E_y=\pm E_{y0}E_x/E_{x0}$,线偏\\
    \indent\textbf{偏振分解}:$\bm{E}=\frac{E_x+jE_y}{\sqrt{2}}\hat{R}+\frac{E_x-jE_y}{\sqrt{2}}\hat{L}$,其中右旋分量$\hat{R}=(\hat{x}-j\hat{y})/\sqrt{2}$,左旋分量$\hat{L}=(\hat{x}+j\hat{y})/\sqrt{2}$\\
    \textbf{TE模($\bm{E}\parallel$界面)在介质界面上的反/折射}:$\phi_1=\phi_{\text{rf}}$,$n\sin\phi_1=n_2\sin\phi_2$(\textbf{Snell定律}),$E_{\text{in}0}=E_{\text{rf}0}=E_{\text{tr}0}$,$n_1\cos\phi_1E_{\text{in}0}-n_2\cos\phi_{\text{rf}}E_{\text{rf}0}=n_2\cos\phi_2E_{\text{tr}0}$\\
    \indent\textbf{反射系数}:$\Gamma_{\perp}=\frac{E_{\text{rf}0}}{E_{\text{in}0}}=\frac{n_1\cos\phi_1-n_2\cos\phi_2}{n_1\cos\phi_1+n_2\cos\phi_2}=\frac{n_1\cos\phi_1-\sqrt{n_2^2-n_1^2\sin^2\phi_1}}{n_1\cos\phi_1+\sqrt{n_2^2-n_1^2\sin^2\phi_1}}$(\textbf{Fresnel方程});\textbf{反射率}:$R_{\perp}=\abs{\Gamma_{\perp}}^2$\\
    \indent若$\perp$入射,$\Gamma_{\perp}=\frac{n_1-n_2}{n_1+n_2}$;\qquad若光疏$\perp$入光密,$\Gamma_{\perp}<0$,入/反射相位差$\pi$\\
    \indent若光密入光疏,$\phi_1>\phi_c=\arcsin\frac{n_2}{n_1}$,则\textbf{全反射},$\phi_2$为复数,$\Gamma_{\perp}=\frac{n_1\cos\phi_1+j\sqrt{n_1^2\sin^2\phi_1-n_2^2}}{n_1\cos\phi_1-j\sqrt{n_1\sin^2\phi_1-n_2^2}}=e^{j2\Phi_{\perp}}$,$\abs{\Gamma_{\perp}}=1$,$R=1$,$\Phi_{\perp}=\arctan\frac{\sqrt{n_1^2\sin^2\phi_1-n_2^2}}{n_1\cos\phi_1}$\\% 图
    \textbf{TM模($\bm{H}\parallel$界面)在介质界面上的反/折射}:\\
    \indent\textbf{反射系数}:$\Gamma_{\parallel}=\frac{H_{\text{rf}0}}{H_{\text{in}0}}=\frac{n_2\cos\phi_1-n_1\cos\phi_2}{n_2\sin\phi_1+n_1\cos\phi_2}=\frac{n_2^2\cos\phi_1-n_1\sqrt{n_2^2-n_1^2\sin^2\phi_1}}{n_2^2\cos\phi_1+n_1\sqrt{n_2^2-n_1^2\sin^2\phi_1}}$\\
    \indent\textbf{布儒斯特角}:若$\phi_1=\phi_B=\arctan\frac{n_2}{n_1}$,$\Gamma_{\perp}=0$,TM全折射,反射仅含TE\\
    \indent若$\phi_1>\phi_c$,$\cos\phi_2=-j\sqrt{(\frac{n_1}{n_2})^2\sin^2\phi_1-1}$,$\Gamma_{\parallel}=\frac{n_2^2\cos\phi_1+jn_1\sqrt{n_1^2\sin^2\phi_1-n_2^2}}{n_2^2\cos\phi_1-jn_1\sqrt{n_1^2\sin^2\phi_1-n_2^2}}=e^{j2\phi_{\parallel}}$,$\abs{\Gamma_{\parallel}}=1$,$\Phi_{\parallel}=\arctan\frac{n_1\sqrt{n_1^2\sin^2\phi_1-n_2^2}}{n_2^2\cos\phi_1}$\\
    \rule{\columnwidth}{.5pt}\\
    \textbf{波导}:默认沿$z$传输,$\bm{E}(\bm{r},\omega)=[\bm{e}_t(x,y)+\hat{z}e_z(x,y)]e^{-j\beta z}$,$\bm{H}(\bm{r},\omega)=[\bm{h}_t(x,y)+\hat{z}e_z(x,y)]e^{-j\beta z}$,其中$\beta$-传播常数\\
    \indent\textcircled{\tiny{1'}}$\Rightarrow-\hat{z}\times\nabla_te_z(x,y)-j\beta\hat{z}\times\bm{e}_t(x,y)=-j\omega\mu_0\bm{h}_t(x,y)$(\textcircled{\tiny{5}})\\
    \indent\indent$\nabla_t\times\bm{e}_t(x,y)=-j\omega\mu_0h_z(x,y)$(\textcircled{\tiny{6}})\\
    \indent\textcircled{\tiny{2'}}$\Rightarrow-\hat{z}\times\nabla_th_z(x,y)-j\beta\hat{z}\times\bm{h}_t(x,y)=j\omega\epsilon_0n^2(x,y)\bm{e}_t(x,y)$(\textcircled{\tiny{7}})\\
    \indent\indent$\nabla_t\times\bm{h}_t(x,y)=j\omega\epsilon_0n^2(x,y)e_z(x,y)\hat{z}$(\textcircled{\tiny{8}})\\
    \indent\textcircled{\tiny{7}}入$\hat{z}\times$\textcircled{\tiny{5}}$\Rightarrow\bm{e}_t(x,y)=\frac{j[\beta\nabla_te_z(x,y)-\omega\mu_0\hat{z}\times\nabla_th_z(x,y)]}{\beta^2-\omega^2\mu_0\epsilon_0n^2(x,y)}$\\
    \indent\textcircled{\tiny{5}}入$\hat{z}\times$\textcircled{\tiny{7}}$\Rightarrow\bm{h}_t(x,y)=\frac{j[\beta\nabla_th_z(x,y)+\omega\epsilon_0n^2(x,y)\hat{z}\times\nabla_te_z(x,y)]}{\beta^2-\omega^2\mu_0\epsilon_0n^2(x,y)}$\\
    \indent式左均横向分量,右均纵向分量\\
    \textbf{平板波导}:不失一般性,沿$y$无限延展,芯层折射率$n_f>$衬底$n_s>$包层$n_c$\\
    \indent\textbf{TE模}:有$e_y$,$h_x$,$h_z$分量\\
    \indent$e_y(y)=\left\{\begin{array}{ll}
        E_ce^{-\gamma_cx},&x>0\\
        E_f\cos(k_fx+\phi)=E_c[\cos k_fh-\frac{\gamma_c}{k_f}\sin k_fx],&-h\leq x\leq 0\\
        E_se^{\gamma_s(x+h)}=E_c[\cos k_fh+\frac{\gamma_c}{\kappa_f}\sin k_fh]e^{\gamma_s(x+h)},&x<-h
    \end{array}\right.$\\
    \indent其中$\gamma_c=\sqrt{\beta^2-k^2n_c^2}$,$k_f=\sqrt{k^2n_f^2-\beta^2}$,$\gamma_s=\sqrt{\beta^2-k^2n_s^2}$,$\because n_c<n_s<n_f$,$\therefore k^2n_c^2<k^2n_s^2<\beta^2<k^2n_f^2$\\
    \indent\textbf{TE特征/色散方程}:$\frac{\mathrm{d}^2e_y}{\mathrm{d}x^2}+[k^2n^2(x)-\beta^2]e_y(x)=0\Rightarrow k_fh=\arctan\frac{\gamma_r}{k_f}+\arctan\frac{\gamma_s}{k_f}+m\pi$\\
    \indent\textbf{TM模}:有$h_y,e_x,e_z$分量\\
    \indent$h_y(x)=\left\{\begin{array}{ll}
        H_ce^{-\gamma_cx},&x>0\\
        H_f\cos(k_fx+\phi)=H_c[\cos k_fx-\frac{n_f^2\gamma_c}{n_c^2k_f}\sin k_fx],&-h\leq x\leq 0\\
        H_se^{\gamma_s(x+h)}=H_c[\cos k_fh+\frac{n_f^2\gamma_c}{n_c^2k_f}\sin k_fh]e^{\gamma_s(x+h)},&x<-h
    \end{array}\right.$\\
    \indent\textbf{TM特征方程}:$\frac{\mathrm{d}}{\mathrm{d}x}[\frac{1}{n^2(x)}\frac{\mathrm{d}h_y}{\mathrm{d}x}]+[k^2-\frac{\beta^2}{n^2(x)}]h_y(x)=0\Rightarrow k_fh=\arctan(\frac{n_f^2}{n_c^2}\frac{\gamma_c}{k_f})+\arctan(\frac{n_f^2}{n_s^2}\frac{\gamma_s}{k_f})+m'\pi$\\
    \textbf{归一化系数}:\textbf{非对称度量}:$a=\frac{n_s^2-n_c^2}{n_f^2-n_s^2}$,表征波导上下非对称性\\
    \indent\textbf{归一化频率/厚度}:$V=kh\sqrt{n_f^2-n_s^2}$\quad\textbf{可导因子}:$b=\frac{N^2-n_s^2}{n_f^2-n_s^2}$,其中\textbf{有效折射率}$N=\frac{\beta}{k}$\\
    \indent$c=\frac{n_s^2}{n_f^2}$,\qquad$d=\frac{n_c^2}{n_f^2}=c-a(1-c)$\\
    \indent通常$n_c<n_s<N<n_f$,$\therefore 0<b<1$,$d<c<1$\\
    \indent$k_fh=V\sqrt{1-b}$,$\gamma_sh=V\sqrt{b}$,$\gamma_ch=V\sqrt{a+b}$\\
    \textbf{归一化TE}:{\tiny$e_y(x)=\left\{\begin{array}{ll}
        E_c\exp(-V\sqrt{a+b}x/h),&x\geq 0\\
        E_c[\cos(\frac{V\sqrt{1-b}x}{h})-\sqrt{\frac{a+b}{1-b}}\sin(\frac{V\sqrt{1-b}x}{h})],&-h\leq x<0\\
        E_c[\cos(V\sqrt{1-b})+\sqrt{\frac{a+b}{1-b}}\sin(V\sqrt{1-b})]e^{V\sqrt{b}[1+(x/h)]},&x<-h
    \end{array}\right.$}\\
    \indent\textbf{归一化TE特征方程}:$V\sqrt{1-b}=\arctan\sqrt{\frac{a+b}{1-b}}+\arctan\sqrt{\frac{b}{1-b}}+m\pi$\\
    \textbf{归一化TM}:{\tiny$h_y(x)=\left\{\begin{array}{ll}
        H_ce^{-V\sqrt{a+b}x/h},&x>0\\
        H_c[\cos\frac{V\sqrt{1-b}x}{h}-\frac{1}{d}\sqrt{\frac{a+b}{1-b}}\sin\frac{V\sqrt{1-b}x}{h}],&-h\leq x\leq 0\\
        H_c[\cos V\sqrt{1-b}+\frac{1}{d}\sqrt{\frac{a+b}{1-b}}\sin V\sqrt{1-b}]e^{V\sqrt{b}[1+x/h]},&x<-h
    \end{array}\right.$}\\
    \indent\textbf{归一化TM特征方程}:$V\sqrt{1-b}=\arctan\frac{1}{d}\sqrt{\frac{a+b}{1-b}}+\arctan\frac{1}{c}\sqrt{\frac{b}{1-b}}+m'\pi$\\
    \textbf{截止频率/厚度}:模式允许存在的最小频率/厚度,$b=0$入特征方程得\\
    \indent对\textbf{TE}有$V_m=m\pi+\arctan\sqrt{a}$,\qquad若$a=0$,$V_m=m\pi$,$h=\frac{m\lambda}{2\sqrt{n_f^2-n_s^2}}$\\
    \indent对\textbf{TM}有$V_{m'}=m'\pi+\arctan\frac{\sqrt{a}}{d}$,\qquad当$a=0$,$V_{m'}=m'\pi$,$h=\frac{m'\lambda}{2\sqrt{n_f^2-n_s^2}}$\\
    \indent若$V\gg 1$,总模式数$\approx 2(1+V/\pi)$\\
    \textbf{$b-V$图特征}:$V\uparrow\Rightarrow b\uparrow$,对应一个$V$或有一个或多个模式($b$);$h$,$(n_f^2-n_s^2)\uparrow$或$\lambda\downarrow$,则$V\uparrow$,模式数$\uparrow$;低阶模$\beta>$高阶模;若$a=0$,基模$b-V$曲线过原点\\
    \textbf{模式计算步骤}:已知波导结构($h$,$n_c$,$n_f$,$n_s$)和模式波长$\lambda$,算$a$,$c$,$d$,$V$,由$b-V$图得$b$,$N$,$\beta$,模场\\
    \textbf{TE能流}:$\bm{S}=\frac{\beta\abs{e_y}^2}{2\omega\mu_0}\hat{z}$,\qquad\textbf{TE单位y上功率}:$P=\frac{\beta}{4\omega\mu_0}E_f[h+\frac{1}{\gamma_s}+\frac{1}{\gamma_c}]=\frac{\beta}{4\omega\mu_0}E_f^2h_{\text{eff}}$\\
    \indent其中\textbf{等效模场厚度}$h_{\text{eff}}=h+\frac{1}{\gamma_s}+\frac{1}{\gamma_c}$\\
    \indent\textbf{归一化模场厚度}:$H=k_fh_{\text{eff}}\sqrt{n_f^2-n_s^2}=V+\frac{1}{\sqrt{a+b}}+\frac{1}{\sqrt{b}}$\\
    \textbf{TE芯层束缚因子}:$\Gamma_f=\frac{\text{芯层传输功率}}{\text{总传输功率}}=\frac{h+\frac{\gamma_c}{k_f^2+\gamma_c^2}+\frac{\gamma_s}{k_f^2+\gamma_s^2}}{h+\frac{1}{\gamma_c}+\frac{1}{\gamma_s}}=\frac{V+\sqrt{b}+\frac{\sqrt{a+b}}{1+a}}{V+\frac{1}{\sqrt{b}}+\frac{1}{\sqrt{a+b}}}$\\
    \indent\textbf{衬底束缚因子}$\Gamma_s=\frac{\text{衬底传输功率}}{\text{总传输功率}}=\frac{1-b}{\sqrt{b}[V+\frac{1}{\sqrt{b}}+\frac{1}{\sqrt{a+b}}]}$\\
    \indent\textbf{包层束缚因子}$\Gamma_c=\frac{\text{包层传输功率}}{\text{总传输功率}}=\frac{1-b}{(1+a)\sqrt{a+b}[V+\frac{1}{b}+\frac{1}{\sqrt{a+b}}]}$\\
    \textbf{TM能流}:$\bm{S}=\frac{\beta\abs{h_y}^2}{2\omega\epsilon_0n(x)^2}\hat{z}$,\textbf{单位y上功率}:$P=\frac{\beta}{4\omega\epsilon_0}\frac{H_f^2}{n_f^2}[h+\frac{1}{\gamma_sq_s}+\frac{1}{\gamma_cq_c}]=\frac{\beta}{4\omega\epsilon_0}\frac{H_f^2}{n_f^2}h_{\text{eff}}$\\
    \indent其中$q_s=\frac{N^2}{n_s^2}+\frac{N^2}{n_f^2}-1$,$q_c=\frac{N^2}{n_c^2}+\frac{N^2}{n_f^2}-1$,\textbf{等效模场厚度}:$h_{\text{eff}}=h+\frac{1}{\gamma_sq_s}+\frac{1}{\gamma_cq_c}$\\
    几何光学角度看,\textbf{导模}:波导界面上全反射;\textbf{辐射模}:波导界面上有折射\\
    \textbf{相速度}:等相位面移速,$v_p=\frac{\omega}{\beta}=\frac{\omega}{kN}=\frac{c}{N}$,其中$N$-等效折射率,高阶模相速大\\
    \textbf{群速度}:波包移速,本质是介质对非单色光的色散,$v_g=\frac{\mathrm{d}\omega}{\mathrm{d}\beta}=\frac{c}{n_g}$,其中群折射率$n_g=\frac{\mathrm{d}\beta}{\mathrm{d}k}$\\
    \textbf{相/群速关系}:$\frac{c^2}{v_pv_g}=n_f^2(b+\frac{V}{2}\frac{\mathrm{d}n}{\mathrm{d}V})+n_s^2(1-b-\frac{V}{2}\frac{\mathrm{d}b}{\mathrm{d}V})$,\qquad低阶模群速度大\\
    \indent对\textbf{TE模},$\frac{c^2}{v_pv_g}=n_f^2\Gamma_f+n_s^2\Gamma_s+n_c^2\Gamma_c$\\
    \indent对良好束缚(well-guided)的波导,能量主要束缚在芯层,$\Gamma_f\approx 1$,$\Gamma_s\approx\Gamma_c\approx 0\Rightarrow\frac{c^2}{v_pv_g}\approx n_f^2$\\
    \rule{\columnwidth}{.5pt}\\
    \textbf{波导传输损耗}:来源:1)光与介质中电子(主要),原子,分子相互作用致吸收损耗,化为热,声,2)波导结构缺陷,包括几何上的不规则,材料缺陷和不均匀,(对玻璃等无定型材料)团簇大小和组分的涨落,致散射损耗,表现为反向传播,跳模,辐射模\\
    \textbf{复电极化率}:$\nabla\times\bm{H}=(j\omega\epsilon+\sigma)\bm{E}=j\omega\epsilon_0\tilde{\epsilon}_r\bm{E}\Rightarrow\tilde{\epsilon}_{r}=\frac{\epsilon}{\epsilon_0}-j\frac{\omega}{\epsilon}=\epsilon_r-j\epsilon_i$\\
    \textbf{Drude(/自由电子)模型}(适用含大量无束缚载流子的介质):$\tilde{\epsilon}_r=1-\frac{\omega_p^2}{\omega^2+\omega_c^2}-j\frac{\omega_c\omega_p^2}{\omega(\omega^2+\omega_c^2)}$,其中$\omega_c$-碰撞频率,$\omega_p$-等离子体频率;对金属,自由电子罕碰撞,$\omega_c=0$,$\epsilon_i=0$,$\tilde{\epsilon}_r=1-(\frac{\omega_p}{\omega})^2$\\
    \textbf{Lorenz模型}(适用电荷受核束缚的介质):$\tilde{\epsilon}_r=1-\frac{\omega_p(\omega^2-\omega_0^2)}{(\omega^2-\omega_0^2)+\omega^2\omega_c^2}-j\frac{\omega_p^2\omega_c\omega}{(\omega^2-\omega_0^2)+\omega^2\omega_c^2}$,其中$\omega$-谐振频率\\
    \textbf{复折射率}:$\tilde{n}=\sqrt{\tilde{\epsilon}_r}=n-j\kappa$,其中$n=(\frac{\epsilon_r+\sqrt{\epsilon_r^2+\epsilon_i^2}}{2})^{1/2}$,$\kappa=(\frac{-\epsilon_r+\sqrt{\epsilon_r^2+\epsilon_i^2}}{2})^{1/2}$,通常(半导体,绝缘体等)$\kappa\ll n$,对金属$\kappa\gg n$;\textbf{复波矢}:$\tilde{k}=k\tilde{n}=nk-j\kappa k\Rightarrow$$\abs{E}\propto\abs{e^{j\omega t-j\tilde{k}x}}=e^{-\kappa kx}$;\textbf{衰减系数}$\alpha=\kappa k$,\textbf{衰减长度(集肤深度)}:$\alpha^{-1}=(\kappa k)^{-1}$,对平面波导,$\tilde{n}_c=n_c-j\kappa_c$,$\tilde{n}_f=n_f-j\kappa_f$,$\tilde{n}_s=n_s-j\kappa_s$,对TE模,$\alpha_{\text{TE}}=k[\kappa_sn_s\int_{-\infty}^{-h}\abs{e_y(x)}^2\,\mathrm{d}x+\kappa_fn_f\int_{-h}^0\abs{e_y(x)}^2\,\mathrm{d}x+\kappa_cn_c\int_0^{+\infty}\abs{e_y(x)}^2\,\mathrm{d}x]/[N\int_{-\infty}^{+\infty}\abs{e_y(x)}^2\,\mathrm{d}x]$\\
    \textbf{金属包层平板波导}:$\because$完美导体内无电场,由边界条件$e_y(0)=0\forall$TE;TM有少量$h_y(x)$渗入金属,损耗$>$TE;TM$_0$能量大量集中于与金属交界面附近,称\textbf{表面波};$\tilde{\beta}=\beta-j\alpha$,对良好束缚波导,$\text{b}\approx 1\Rightarrow\beta\approx n_fk$,$\tilde{k}_f=\sqrt{k^2\tilde{n}_f^2-\tilde{\beta}^2}\approx 0$,$\tilde{\gamma}_c=\sqrt{\tilde{\beta}^2-k^2\tilde{n}_c^2}$,$\abs{\tilde{\gamma}_c}\gg\abs{\tilde{k}_f}$,$\arctan\frac{\tilde{\gamma}_c}{\tilde{k}_f}\approx\frac{\pi}{2}-\arctan\frac{\tilde{k}_f}{\tilde{\gamma}_c}\approx\frac{\pi}{2}-\frac{\tilde{k}_f}{\tilde{\gamma}_c}$,$\tilde{\gamma}_s=\sqrt{\tilde{\beta}^2-k^2\tilde{n}_s^2}$,$\abs{\tilde{\gamma}_s}\gg\abs{\tilde{k}_f}$,$\arctan\frac{\tilde{\gamma}_s}{\tilde{k}_f}\approx\frac{\pi}{2}-\frac{\tilde{k}_f}{\tilde{\gamma}_s}$,\textbf{TE特征方程}:$\tilde{k}_fh\approx(m+1)\pi-\frac{\tilde{k}_f}{\tilde{\gamma}_c}-\frac{\tilde{k}_f}{\tilde{\gamma}_s}\Rightarrow\tilde{k}_f=\frac{(m+1)\pi}{h}(1+\frac{1}{\tilde{\gamma}_ch}+\frac{1}{\tilde{\gamma}_sh})^{-1}\Rightarrow\tilde{\beta}_{\text{TE}m}=\sqrt{k^2\tilde{n}_f^2-\tilde{k}_f^2}\approx k\tilde{n}_f(1-\frac{\tilde{k}_f^2}{2k^2\tilde{n}_f^2})\approx k\tilde{n}_f-\frac{(m+1)^2\pi^2}{2k\tilde{n}_fh^2}(1+\frac{1}{\tilde{\gamma}_sh}+\frac{1}{\tilde{\gamma}_ch})^{-2}$,若芯层无损,$\kappa_f=0$,$\tilde{n}_f=n_f$,$\frac{\tilde{\beta}_{\text{TE}m}}{k}\approx n_f-\frac{(m+1)^2\pi}{2n_f(kh)^2}(1+\frac{1}{kh\sqrt{n_f^2-\tilde{n}_c^2}}+\frac{1}{kh\sqrt{n_f^2-\tilde{n}_s^2}})$,$\frac{\alpha_{\text{TE}m}}{k}\approx\frac{(m+1)^2\pi^2}{2n_f(kh)^2}\im[\frac{1}{kh\sqrt{n_f^2-\tilde{n}_c^2}}+\frac{1}{kh\sqrt{n_f^2-\tilde{n}_s^2}}]^{-2}$,$\because$通常$\abs{\epsilon_r}\gg\epsilon_i$,$\therefore\frac{\alpha_{\text{TE}m}}{k}\approx\frac{(m+1)^2\pi^2}{2n_f(kh)^2}\im[-2(\frac{1}{\sqrt{n_f^2-\epsilon_{cr}+j\epsilon_{ci}}}+\frac{1}{\sqrt{n_f^2-\epsilon_{sr}+j\epsilon_{si}}})]\approx\frac{(m+1)^2\pi^2}{2n_f(kh)^2}[\frac{\epsilon_{ci}}{(n_f^2-\epsilon_{cr})^{3/2}}+\frac{\epsilon_{si}}{(n_f^2-\epsilon_{sr})^{3/2}}]$;\textbf{TM}同理$\frac{\tilde{\beta}_{\text{TM}m'}}{k}\approx n_f-\frac{(m'+1)^2\pi^2}{2n_f(kh)^2}[1+\frac{\tilde{n}_c^2}{n_f}\frac{1}{kh\sqrt{n_f^2-\tilde{n}_c^2}}+\frac{\tilde{n}_s}{n_f^2}\frac{1}{kh\sqrt{n_f^2-\tilde{n}_s^2}}]^{-2}$,$\frac{\alpha_{\text{TM}m'}}{k}\approx\frac{(m'+1)^2\pi^2}{2n_f(kh)^2}[\frac{\epsilon_{ci}(2n_f^2-\epsilon_{cr})}{n_f^2(n_f^2-\epsilon_{cr})^{3/2}}+\frac{\epsilon_{si}(2n_f^2-\epsilon_{sr})}{n_f^2(n_f^2-\epsilon_{sr})^{3/2}}]$;$m\uparrow$,$h\uparrow$,则$\alpha\downarrow$;$\because\frac{2n_f^2-\epsilon_{cr/sr}}{n_f^2}>1$,$\therefore$同阶TE损耗$<$TM;对包层$\backslash$衬底均金属的TM$_0$,$n_s^2=n_c^2=\epsilon_1$,$n_f^2=\epsilon_2$,由麦氏方程,$\tilde{\beta}=k\sqrt{\frac{\epsilon_1\epsilon_2}{\epsilon_1+\epsilon_2}}\Rightarrow N^2=\frac{\epsilon_1\epsilon_2}{\epsilon_1+\epsilon_2}\Rightarrow\re(\frac{N^2}{n_f^2})=\re(\frac{\epsilon_1}{\epsilon_1+\epsilon_2})>1$,或由特征方程,$\tilde{k}_fh=2\arctan\frac{n_f^2}{\tilde{n}_s^2}\frac{\tilde{\gamma}_s}{\tilde{k}_f}+m'\pi$,其中$\tilde{k}_f=j\sqrt{\tilde{\beta}^2-k^2n_f^2}$,$j\sqrt{\tilde{\beta}^2-k^2n_f^2}h=m'\pi-j2\arctanh\frac{n_f^2}{\tilde{n}_s^2}\frac{\sqrt{\tilde{\beta}^2-k^2\tilde{n}_s^2}}{\sqrt{k^2n^2-\tilde{\beta}^2}}$,金属$\abs{\epsilon_{sr}}\gg\epsilon_{si}$,$\therefore\tilde{n}_s^2=\epsilon_{sr}-j\epsilon_{si}\approx\re[\tilde{n}_s^2]<0\Rightarrow j\sqrt{\tilde{\beta}^2-k^2n_f^2}h=m'\pi-j\arctanh\frac{n_f^2}{\re[\tilde{n}_s^2]}\frac{\sqrt{\tilde{\beta}^2-k^2\re[\tilde{n}_s^2]}}{\sqrt{\tilde{\beta}^2-k^2n_f^2}}$,对$m'\neq 0$,式左纯虚,$\therefore\beta$必非纯实,对$m'=0$,$\tanh\frac{\sqrt{\tilde{\beta}^2-k^2n_f^2}h}{2}=-\frac{n_f^2}{\tilde{n}_s^2}\frac{\sqrt{\tilde{\beta}^2-k^2\tilde{n}_s^2}}{\sqrt{\tilde{\beta}^2-k^2n_f^2}}$,良好束缚时$\tilde{k}_fh\rightarrow\infty$,$\therefore-\frac{n_f^2}{\tilde{n}_s^2}\frac{\sqrt{\tilde{\beta}^2-k^2\tilde{n}_s^2}}{\sqrt{\tilde{\beta}^2-k^2n_f^2}}\approx 1\Rightarrow\frac{\tilde{\beta}}{k}\approx\sqrt{\frac{n_f^2\tilde{n}_s^2}{n_f^2+\tilde{n}_s^2}}$,沿芯层与金属交界面附近传播,并非由折射率束缚,称表面等离子基元波\\% 图
    \rule{\columnwidth}{.5pt}\\
    \textbf{3D波导}:模式命名:$E_{p,q}^{x/y}$,其中$x/y$-主要电场分量方向,$p-1$,$q-1$-$x$,$y$方向电场分布零点数\\
    \textcircled{\tiny{5}}$\Rightarrow-j\beta\hat{z}\times[e_x(x,y)\hat{x}+e_y(x,y)\hat{y}]-\hat{z}\times[\frac{\partial e_x(x,y)}{\partial x}\hat{x}+\frac{\partial e_y(x,y)}{\partial y}\hat{y}]$,\textcircled{\tiny{6}}$\Rightarrow\hat{z}[\frac{\partial e_y(x,y)}{\partial x}-\frac{\partial e_x(x,y)}{\partial y}]=-j\omega\mu_0h_z(x,y)\hat{z}$\\
    弱导条件(weakly guiding,$n_f\approx n_s$,3D波导通常用衬底掺杂实现,折射率变化很小,故适用,与良好束缚不冲突)下,$k_f^2=k^2n_f^2-\beta^2=k^2n_f(n_f+n_s)(1-b)\Delta\approx 2k^2n_f^2(1-b)\Delta\Rightarrow\frac{k_f}{kn_f}=\sqrt{2}\sqrt{(1-b)\Delta}<\sqrt{2\Delta}\sim o(\delta)$,其中$\Delta=\frac{n_f-n_s}{n_f}$,$o(\delta)$-一阶小量,$k_f^2=k_x^2+k_x^2\Rightarrow\frac{k_{x/y}}{kn_f}\sim\delta$;对良好束缚的$E^y$模,$\abs{H_x}\sim\frac{n}{\eta_0}\abs{E_y}\sim o(1)$,$\abs{H_z}\sim\frac{n}{\eta_0}\abs{E_z}\sim o(\delta)$,$\abs{H_z}\sim\frac{n}{\eta_0}\abs{E_x}\sim o(\delta^2)$,$\frac{n}{\eta_0}E_x=o(\delta^2)$,$\frac{n}{\eta_0}E_y=-\frac{\beta}{kn}H_x+o(\delta^2)=-\frac{kn}{\beta}H_x+o(\delta^2)$,$\frac{n}{\eta_0}E_z=\frac{j}{kn}\frac{\partial H_x}{\partial y}+o(\delta^2)$,$H_y=o(\delta^2)$,$H_z=-\frac{j}{\beta}\frac{\partial H_x}{\partial x}+o(\delta^2)$;证:初始有$\abs{E_y}\sim 1$,故$H_y$可忽略,\textcircled{\tiny{3'}}$\Rightarrow\frac{\partial H_x}{\partial x}+\frac{\partial H_z}{\partial z}=\frac{\partial H_x}{\partial x}-j\beta H_z=0\Rightarrow\abs{H_z}\sim\abs{\frac{1}{\beta}\frac{\partial H_x}{\partial x}}\sim\abs{\frac{k_x}{\beta}H_z}\sim\abs{\frac{k_x}{nk}H_x}\sim\delta$,\textcircled{\tiny{2'}}$\Rightarrow\frac{\partial H_x}{\partial z}-\frac{\partial H_z}{\partial x}=j\beta H_x-\frac{\partial H_z}{\partial x}=j\omega\epsilon_0n^2E_y\Rightarrow\frac{n}{\eta_0}E_y=\frac{j}{kn}\frac{\partial H_z}{\partial x}-\frac{\beta}{kn}H_x$,其中$\abs{\frac{1}{kn}\frac{\partial H_z}{\partial x}}\sim\abs{\frac{k_x}{kn}H_z}\sim\delta^2\Rightarrow\abs{H_x}\sim\abs{\frac{\beta}{kn}H_x}\sim\abs{\frac{n}{\eta_0}E_y}\sim o(1)$,$H_z\approx-\frac{j}{\beta}\frac{\partial H_x}{\partial x}$入$j\beta H_x-\frac{\partial H_z}{\partial x}=j\omega\epsilon_0n^2E_y\Rightarrow\frac{n}{\eta_0}E_y\approx\frac{1}{kn\beta}(\frac{\partial^2H_x}{\partial x}-\beta^2H_x)$,$\nabla_t^2H_x+(k^2n^2-\beta)H_x=0$入$\Rightarrow\frac{n}{\eta_0}E_y\approx-\frac{1}{kn\beta}(\frac{\partial^2H_z}{\partial y^2}+k^2n^2H_x)$,其中$\abs{\frac{1}{kn\beta}\frac{\partial^2H_z}{\partial y^2}}\sim\abs{\frac{k_y^2}{k^2n^2}H_x}\sim o(\delta)^2\Rightarrow\frac{n}{\eta_0}E_y\approx-\frac{kn}{\beta}H_x$,\textcircled{\tiny{2'}}$\Rightarrow j\omega\epsilon_0n^2E_y\approx\frac{\partial H_y}{\partial y}\Rightarrow\frac{n}{\eta_0}E_x\approx-\frac{j}{kn}\frac{\partial H_z}{\partial y}\Rightarrow\abs{\frac{n}{\eta_0}E_x}\sim o(\delta^2)$,\textcircled{\tiny{2'}}$\Rightarrow j\omega\epsilon_0n^2E_z\approx\frac{\partial H_x}{\partial y}\Rightarrow\frac{n}{\eta_0}E_z\approx\frac{j}{kn}\frac{\partial H_x}{\partial y}\Rightarrow\abs{\frac{n}{\eta_0}E_z}\sim\abs{\frac{k_y}{kn}H_x}\sim o(\delta)$,\textcircled{\tiny{1'}}$\Rightarrow-j\omega\mu_0 H_y=\frac{\partial E_x}{\partial z}-\frac{\partial E_z}{\partial x}\Rightarrow H_y=\frac{\beta}{\omega\mu_0}E_x-\frac{j}{\omega\mu_0}\frac{\partial E_z}{\partial x}\approx\frac{n}{\eta_0}E_x-\frac{j}{kn}\frac{n}{\eta_0}\frac{\partial E_z}{\partial x}\Rightarrow\abs{H_y}\sim o(\delta^2)$;对良好束缚的$E^x$模,$\abs{H_y}\sim\frac{n}{\eta_0}\abs{E_x}\sim o(1)$,$\abs{H}_z\sim\frac{n}{\eta_0}\abs{E_z}\sim o(\delta)$,$\abs{H}_x\sim\frac{n}{\eta_0}\abs{E_y}\sim o(\delta^2)$,$\frac{n}{\eta_0}E_x=\frac{\beta}{kn}H_y+\delta(\delta^2)=\frac{kn}{\beta}H_y+o(\delta^2)$,$\frac{n}{\eta_0}E_y=o(\delta^2)$,$\frac{n}{\eta_0}E_y=o(\delta^2)$,$\frac{n}{\eta_0}E_z=-\frac{j}{kn}\frac{\partial H_y}{\partial x}+o(\delta^2)$,$H_x=o(\delta^2)$,$H_z=-\frac{-j}{\beta}\frac{\partial H_y}{\partial y}+o(\delta^2)$\\
    \textbf{Marcatili方法}:将3D波导$n(x,y)=n_1(\text{R1}:\abs{x}\leq\frac{w}{2},\abs{y}\leq\frac{h}{2}),n_2(\text{R2}:\abs{x}\leq\frac{w}{2},y>\frac{h}{2}),n_3(\text{R3}:x>\frac{w}{2},\abs{y}\leq\frac{h}{2}),n_4(\text{R4}:\abs{x}\leq\frac{w}{2},y<\frac{h}{2}),n_5(\text{R5}:x<-\frac{w}{2},\abs{y}\leq\frac{h}{2})$拆解为横向平板波导H$n(y)=n_1(\abs{y}\leq\frac{h}{2}),n_2(y>\frac{h}{2}),n_4(y<-\frac{h}{2})$和纵向平板波导W$n(x)=n_1(\abs{x}\leq\frac{w}{2}),n_3(x>\frac{w}{2}),n_5(x<-\frac{w}{2})$分别求解;对$E^y$模,R1有$H_{x1}=C_1\cos(k_{x1}x+\phi_{x1})\cos(k_{y1}+\phi_{y1})e^{-j\beta z}$,R2有$H_{x2}=C_2\cos(k_{x2}x+\phi_{x2})e^{-jk_{y2}y}e^{-j\beta z}$,R3有$H_{x3}=C_3e^{-jk_3x}\cos(k_{y3}y+\phi_{y3})e^{-j\beta z}$,R4有$H_{x4}=C_4\cos(k_{x4}x+\phi_{x4})e^{jk_{y4}y}e^{-j\beta z}$,R5有$H_{x5}=C_5e^{jk_{x5}x}\cos(k_{y5}y+\phi_{y5})e^{-j\beta z}$,其余4角能量少,故可忽略,其中$k_{xj}^2+k_{yj}=\beta^2=k^2n_j^2$,在$y=\pm\frac{h}{2}$,$H_{x1}=H_{x2/4}$,$\Rightarrow k_{x1}=k_{x2}=k_{x4}=k_x$,$\phi_{x1}=\phi_{x2}=\phi_{x4}=\phi_x$,$\frac{n}{\eta_0}E_z\approx\frac{j}{kn}\frac{\partial H_x}{\partial y}\Rightarrow\frac{1}{n^2}\frac{\partial H_x}{\partial y}$连续,$H_z\approx\frac{-j}{\beta}\frac{\partial H_x}{\partial x}\Rightarrow\frac{\partial H_x}{\partial x}$连续,在$x=\pm\frac{w}{2}$,$\mu_0H_{x1}=\mu_0H_{x3/5}\Rightarrow k_{y1}=k_{y3}=k_{y5}$,$\phi_{y1}=\phi_{y3}=\phi_{y5}=\phi_y$,$\frac{n}{\eta_0}E_y\approx-\frac{kn}{\beta}H_x\Rightarrow H_x$连续,$H_z\approx\frac{-j}{\beta}\frac{\partial H_x}{\partial x}\Rightarrow\frac{\partial H_x}{\partial x}$连续,$\frac{n}{\eta_0}E_z\approx\frac{j}{kn}\frac{\partial H_x}{\partial y}\Rightarrow E_{z1}-E_{z3}\approx\frac{j\eta_0}{k}\frac{1}{n_1^2}\frac{\partial}{\partial y}(H_{x1}-H_{x3})-\frac{j\eta_0}{n_3}\msout{\frac{n_1^2-n_3^2}{n_1^2}}o(\delta)\msout{\frac{1}{kn_3}\frac{\partial H_{x3}}{\partial y}}o(\delta)\Rightarrow H_x$连续(已有),在$y=h/2$,$C_1\cos(k_{y}\frac{h}{2}+\phi_y)=C_2e^{-jk_{y2}h/2}$,$-\frac{k_y}{n_1^2}C_1\sin(k_y\frac{h}{2}+\phi_y)=-\frac{jk_{y2}}{n_2^2}C_2e^{-jk_{y2}h/2}$,两式相除$\Rightarrow\tan(k_y\frac{h}{2}+\phi_y)=\frac{jk_{y2}n_1^2}{k_yn_2^2}$,由$k_{xj}^2+k_{yj}^2+\beta^2=k^2n_j^2$,$j=1,2$相减$\Rightarrow jk_{y2}=\sqrt{k^2(n_1^2-n_2^2)-k_y^2}$,回代$\Rightarrow\tan(k_y\frac{h}{2}+\phi_y)=\frac{n_1^2\sqrt{k^2(n_1^2-n_2^2)-n_y^2}}{n_2^2k_y}\Rightarrow$特征方程$k_y\frac{h}{2}+\phi_y=q'\pi+\arctan\frac{n_1^2\sqrt{k^2(n_1^2-n_2^2)-n_y^2}}{n_2^2k_y}$,在$y=-\frac{h}{2}$同理有特征方程$k_y\frac{h}{2}-\phi_y=q''\pi+\arctan\frac{n_1^2\sqrt{k^2(n_1^2-n_4^2)-k_y^2}}{n_4^2k_y}$,两特征方程相加消$\phi_y\Rightarrow k_yh=q\pi+\arctan\frac{n_1^2\sqrt{k^2(n_1^2-n_2^2)-k_y^2}}{n_2^2k_y}+\arctan\frac{n_1^2\sqrt{k^2(n_1^2-n_4^2)}}{n_4^2k_y}$,同理在$x=\pm\frac{w}{2}$,$k_xw=p\pi+\arctan\frac{\sqrt{k^2(n_1^2-n_3^2)-k_x^2}}{k_x}+\arctan\frac{\sqrt{k^2(n_1^2-n_5^2)-k_x^2}}{k_x}$,其中$\beta^2=n_1^2k^2-k_x^2-k_y^2$\\
    \textbf{归一化}:不失一般性,$n_1>n_5>n_4>n_2$,$n_5>n_3$,$V_H=kh\sqrt{n_1^2-n_4^2}$,对H,$a_H=\frac{n_4^2-n_2^2}{n_1^2-n_4^2}$,$b_H=\frac{\beta_H^2-k^2n_4^2}{k^2(n_1^2-n_4^2)}=\frac{N_H^2-n_4^2}{n_1^2-n_4^2}$,$c_H=\frac{n_4^2}{n_1^2}$,$d_H=c_H-a_H(1-c_H)=\frac{n_2^2}{n_1^2}$;对W,$V_W=kw\sqrt{n_1^2-n_5^2}$,$a_w=\frac{n_5^2-n_3^2}{n_1^2-n_5^2}$,$b_W=\frac{\beta_W^2-k^2n_5^2}{k^2(n_1^2-n_5^2)}$\\
    \textbf{计算步骤}:分别由H和W的$b-V$曲线得$b_H$,$b_W\Rightarrow\beta_H$,$\beta_W\Rightarrow k_y^2=n_1^2k^2-\beta_H^2$,$k_x^2=n_1^2-\beta_W^2\Rightarrow\beta^2=n_1k^2-k_x^2-k_y^2-n_1k^2=k^2(n_4^2+n_5^2-n_1^2)+b_Wk^2(n_1^2-n_5^2)+b_Hk^2(n_1^2-n_4^2)$,总传播常数$b_M=\frac{\beta^2-k^2n_5^2}{k^2(n_1^2-n_5^2)}=b_W+\frac{n_1^2-b_4^2}{n_1^2-b_5^2}(b_H-1)$\\
    \textbf{有效折射率法}:类似M法将3D波导拆解为平板波导I(纵向,安排I)/I'(横向,安排II)和II(横向)/II'(纵向),先解I/I'得有效折射率$n_{\text{eff}}^{/'}$(通常$n_{\text{eff}}\neq n_{\text{eff}}'$),将$n_{\text{eff}}^{/'}$作II/II'芯层折射率,得II/II'传播常数$\beta$作为总传播常数;解释:对弱导$E_y$模,$H_x=h_x(x,y)e^{-j\beta z}$,入波动方程$(\nabla^2+k^2n^2)H_x=0\Rightarrow[\frac{\partial^2}{\partial x^2}+\frac{\partial^2}{\partial y^2}+k^2n^2-\beta^2]h_x=0$,分离变量$n_{\text{ps}}^2=n_x^2(x)+n_y^2(y)$,$h_x(x,y)=X(x)Y(y)$,回代$\Rightarrow\frac{1}{X}\frac{\mathrm{d}^2X}{\mathrm{d}x^2}+\frac{1}{Y}\frac{\mathrm{d}^2Y}{\mathrm{d}y^2}+[k^2n_x^2(x)+k^2n_y^2(y)-\beta^2]=0\Rightarrow$\textbf{安排I}:$\frac{1}{Y}\frac{\mathrm{d}^2Y}{\mathrm{d}y^2}+k^2n_y^2=-\frac{1}{X}\frac{\mathrm{d}^2X}{\mathrm{d}x^2}-[k^2n_x^2(x)-\beta^2]\overset{\text{def}}{=}(ke_{\text{eff}})^2\Rightarrow\frac{1}{Y}\frac{\mathrm{d}^2Y}{\mathrm{d}y^2}+k^2[n_y(y)^2-n_{\text{eff}}^2]=0$,$\frac{1}{X}\frac{\mathrm{d}^2X}{\mathrm{d}x^2}+k^2[n_x^2(x)+n_{\text{eff}}^2]-\beta^2=0$,近似为赝3D波导$n_{\text{ps}}^2=n_1^2(\text{R1}),n_2^2(\text{R2}),n_3^2+n_1^2-n_{\text{eff}}^2(\text{R3}),n_4^2(\text{R4}),n_5^2+n_1^2-n_{\text{eff}}^2(\text{R5})$,拆解为横向平板波导$n_y^2(y)=n_1^2(\abs{y}\leq\frac{h}{2}),n_2(y>\frac{h}{2}),n_4(x<-\frac{h}{2})$和纵向平板波导$n_x^2(x)=0(\abs{x}\leq\frac{w}{2}),n_3^2-n_{\text{eff}}^2(x>\frac{w}{2}),n_5^2-n_{\text{eff}}^2(x<-\frac{w}{2})$,在$y=\pm\frac{h}{2}$,$Y$,$\frac{1}{n_y^2}\frac{\mathrm{d}Y}{\mathrm{d}y}$连续,$\Rightarrow kh\sqrt{n_1^2-n_{\text{eff}}^2}=q\pi+\arctan\frac{n_1^2}{n_2^2}\frac{\sqrt{n_{\text{eff}}^2-n_2^2}}{\sqrt{n_1^2-n_{\text{eff}}^2}}+\arctan\frac{n_1^2}{n_4^2}\frac{\sqrt{n_{\text{eff}}^2-n_4^2}}{\sqrt{n_1^2-n_{\text{eff}}^2}}$,同理在$x=\pm\frac{w}{2}$,$X$,$\frac{\mathrm{d}X}{\mathrm{d}x}$连续,$kw\sqrt{n_{\text{eff}}^2-N^2}=p\pi+\arctan\frac{\sqrt{N^2-n_3^2}}{\sqrt{n_{\text{eff}}^2-N^2}}+\arctan\frac{\sqrt{N^2-n_5^2}}{\sqrt{n_{\text{eff}}^2-N^2}}$,其中$N$-3D波导总有效折射率,总传播常数$\beta=kN$,或\textbf{安排II},$\frac{1}{X}\frac{\mathrm{d}^2X}{\mathrm{d}x^2}+k^2n_x^2(x)=-\frac{1}{Y}\frac{\mathrm{d}^2Y}{\mathrm{d}y^2}-[k^2n_y^2(y)-\beta^2]\overset{\text{def}}{=}(kn_{\text{eff}}')^2\Rightarrow\frac{1}{X}\frac{\mathrm{d}^2X}{\mathrm{d}x^2}+k^2[n_x^2(x)-n_{\text{eff}}'^2]=0$,$\frac{1}{Y}\frac{\mathrm{d}^2Y}{\mathrm{d}y^2}+k^2[n_y^2(y)+n_{\text{eff}}'^2]-\beta^2=0$,近似为赝3D波导$n_{\text{sp}}^2=n_1(\text{R1}),n_2^2+n_1^2-n_{\text{eff}}'^2(\text{R2}),n_3^2(\text{R3}),n_3^2(\text{R3}),n_4^2+n_1^2-n_{\text{eff}}'^2(\text{R4}),n_5^2(\text{R5})$,拆解为纵向平板波导$n_x^2(x)=n_1^2(\abs{x}\leq\frac{w}{2}),n_3^2(x>\frac{w}{2}),n_5^2(x<-\frac{w}{2})$和横向平板波导$n_y(y)=0(\abs{y}\leq\frac{h}{2}),n_2^2-n_{\text{eff}}'^2(y>\frac{h}{2}),n_4^2-n_{\text{eff}}'^2(y<-\frac{h}{2})$,同理$\Rightarrow kw\sqrt{n_1^2-n_{\text{eff}}'^2}=p\pi+\arctan\frac{\sqrt{n_{\text{eff}}'^2-n_3^2}}{\sqrt{n_1^2-n_{\text{eff}}'^2}}+\arctan\frac{\sqrt{n_{\text{eff}}'^2-n_5^2}}{\sqrt{n_1^2-n_{\text{eff}}'^2}}$,$kh\sqrt{n_{\text{eff}}'^2-N^2}=q\pi+\arctan\frac{n_{\text{eff}'}^2}{n_2^2}\frac{\sqrt{N^2-n_2^2}}{\sqrt{n_{\text{eff}}'^2-N^2}}+\arctan\frac{n_{\text{eff}}'^2}{n_4^2}\frac{\sqrt{N^2-n_4^2}}{\sqrt{n_{\text{eff}}'^2-N^2}}$\\
    \textbf{计算步骤}:对安排I,波导I,由$b_I-V_I$曲线得可导因子$b_I$,$n_{\text{eff}}^2=n_4^2+b_I(n_1^2-n_2^2)$,对波导II,由$b_{II}-V_{II}$曲线得$b_{II}$,总有效折射率$N^2=n_5^2+b_{II}(n_{\text{eff}}^2-n_5^2)=n_5^2+b_{II}[n_4^2-n_5^2+b_I(n_1^2-n_4^2)]$,总可导因子$b_{KT}=\frac{N^2-n_5^2}{n_1^2-n_5^2}=b_{II}+\frac{n_1^2-n_4^2}{n_1^2-n_5^2}b_{II}(b_I-1)$\\
    \textbf{折射率偏差$\Delta(n^2)$所致$\beta^2$偏差}:$\delta(\beta^2)=\frac{k^2\iint\abs{E(x,y,z)}^2\Delta[n^2(x,y)]\,\mathrm{d}x\,\mathrm{d}y}{\iint\abs{E(x,y,z)}^2\,\mathrm{d}x\,\mathrm{d}y}$,设$n=n_2(\text{R2345})$,对有效折射率法$\Delta(n^2)=n_1^2-n_{\text{eff}}^2(>0,\text{R35}),n_2^2-n_{\text{eff}}^2(\text{4角}),0(\text{其他})$,R35高估,4角低估折射率,R35处能量多于4角,故总体高估折射率,$\delta(\beta^2)>0$;对M法,折射率等效为$n_{\text{eq}}^2(x,y)=n'^2(x)+n''^2(y)$,其中$n'^2(x)=\frac{n_1^2}{2}(\abs{x}\leq\frac{w}{2}),n_2^2-n_1^2/2(x>\frac{w}{2}),n_2^2-n_1^2/2(x<-\frac{w}{2})$,$y''^2(y)=n_1^2/2(\abs{y}\leq\frac{h}{2}),n_2^2-n_1^2/2(y>\frac{h}{2}),n_2^2-n_1^2/2(y<-\frac{h}{2})$,$\Delta(n^2)=n_2^2-n_1^2(<0,\text{4角}),0(\text{其他})$,4角低估折射率,$\delta(\beta^2)<0$
\end{multicols}
\end{document}